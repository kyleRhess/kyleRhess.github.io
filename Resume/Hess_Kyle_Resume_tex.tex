\documentclass[]{article} 
\usepackage{helvetica} % uses helvetica postscript font (download helvetica.sty)
% \usepackage{newcent}   % uses new century schoolbook postscript font
\usepackage[T1]{fontenc}
\usepackage{setspace}
\usepackage{lingmacros}
\usepackage{titlesec}
\usepackage{hyperref}


\thispagestyle{empty}

\usepackage[a4paper,total={7in, 10in}]{geometry}

\titleformat{\section}
{\normalfont\fontsize{11}{13}\bfseries}{\thesection}{1em}{}[{\titlerule[0.5pt]}]

\titlespacing*{\section}{0pt}{2ex plus 2ex}{1ex}

\begin{document}

\begin{singlespace}
\noindent\textbf{\huge{Kyle R. Hess}}
\hfill \break
\noindent\makebox[\linewidth]{\rule{\textwidth}{1pt}}
    \rightline{Huntsville, AL\hspace{3 mm}|\hspace{3 mm}\href{kyleRhess.github.io}{kyleRhess.github.io}\hspace{3 mm}|\hspace{3 mm}kylehess.r@gmail.com}
    \vspace{-8mm}

\section*{Professional Experience}

\subsubsection*{09/2017 - Present\hspace{3 mm}|\hspace{3 mm}Electrical Engineer\hspace{3 mm}|\hspace{3 mm}Gladiator Technologies \& LKD Aerospace}
\textbf{\emph{Hardware \& Firmware}}
\vspace{-2mm}
\begin{itemize}
    \setlength\itemsep{0em}
    \item Work with small teams to develop and debug firmware for IMUs and Inertial Navigation Systems (INS)
    \vspace{-2mm}
    \begin{itemize}
        \setlength\itemsep{0em}
        \item Support new product development innovations as well as existing and deployed firmware
        \item Prototype new hardware, sensors and embedded algorithms to advance our product performance
    \end{itemize}
    \vspace{-1mm}
    \item Create and manage electrical schematics, circuit board layouts, board revisions and bills of materials with OrCAD
    \item Collaborated with a contracted team of experts to write, integrate and test a new 15-state Extended Kalman Filter into a legacy INS/GPS design which resulted in a 10x improvement in "free-inertial" navigation performance
\end{itemize}

\noindent\textbf{\emph{Production Support \& Software Development}}
\vspace{-2mm}
\begin{itemize}
    \setlength\itemsep{0em}
    \item Support our in-house production test and calibration software (a custom C++ Windows applications)
    \vspace{-2mm}
    \begin{itemize}
        \setlength\itemsep{0em}
        \item Maintain legacy testing capabilities while implementing bug fixes and merging new test capabilities
        \item Doubled our potential production capacity by adding support for up to eight units to existing test software
    \end{itemize}

    \vspace{-1mm}
    \item Rewrote our data reduction system using Python to replace cryptic Excel macros and support new products 
    \vspace{-2mm}
    \begin{itemize}
        \setlength\itemsep{0em}
        \item The data reduction scripts extract IMU performance metrics such as noise, bias, scale factor error and linearity
        \item Python allowed for highly flexible data manipulation, coherent code and direct database integration
        \item This transformed a lengthy production process into a one-click solution that removed human error
    \end{itemize}
    \vspace{-1mm}

    \item Took on the roll of a "git champion" to steer our software team towards embracing git as the goto version control system. This resulted in much more efficient code reviews and shorter debugging sessions.
    \item Helped create and use tools for simulating IMU/INS algorithm changes in a post-processing environment

\end{itemize}

\noindent\textbf{\emph{Product Development \& Quality}}
\vspace{-2mm}
\begin{itemize}
    \setlength\itemsep{0em}
    \item Oversee new product development at every stage from concept, qualification and release
    \vspace{-2mm}
    \begin{itemize}
        \setlength\itemsep{0em}
        \item Track all product development progress through phase-gates per AS9100D quality management system
        \item Write work instructions ranging from low-level device assembly to end-item testing procedures
        \item Manage prototype builds from system design to part procurement to final compliance testing 
    \end{itemize}

    \vspace{-1mm}
    \item Own and complete Engineering Change Requests (ECR) as needed to update controlled documents and drawings

\end{itemize}

\noindent\textbf{\emph{Sales Support \& Documentation}}
\vspace{-2mm}
\begin{itemize}
    \setlength\itemsep{0em}
    \item Support customers with application engineering challenges faced during IMU integration (remotely and directly)
    \item Handle all software maintenance and customer questions for our existing Windows SDK package 
    \item Write and maintain product Datasheets, User Guides, Technical Summaries and Reference Manuals 
\end{itemize}

\section*{Skills \& Abilities}

\textbf{Electronics}: Experience with SPI, UART, RS-485 and USB interfaces, 32-bit microcontrollers (STM32 \& NXP K22), schematic design \& capture, PCB layout,
CAD/EDA library management, component selection and soldering
\newline\textbf{Lab Equipment}: Oscilloscopes, DMMs, function generators, power supplies and spectrum analyzers
\newline\textbf{Software}: KiCAD, Visual Studio, Git, Eclipse, OrCAD, Autodesk Eagle, LTSpice, NI Multisim, Smartsheet, AutoCAD
\newline\textbf{Programming}: C (Embedded), C++, Python, MATLAB, C\#, Arduino and System Verilog


\section*{Personal Projects}


\textbf{eBike Motor Driver - Field-Oriented Control (01/2021 - 05/2021)}\newline
Efficient servo driver for a 3-phase brushless DC motor.
Utilizes phase current sensing and incremental encoder feedback to control the 3-phase currents for optimal rotor torque with space vector modulation.
Cascaded PID controllers allow for precise speed and position control with an encoder resolution of 0.26 mrad.\newline
This motor drive was then successfully used as the controller for a 1.5 kW eBike hub motor.
\vspace{-2mm}\newline\newline
\textbf{Racing drone flight controller (12/2019)}\newline
A custom controller built around an ARM Cortex M-4. Utilizes an IMU, barometer, and GPS receiver. A PID loop
running at 1 kHz controls four motors for stabilization and flight control.


\section*{Education}

\textbf{University of Washington, Seattle, WA\hspace{3 mm}|\hspace{3 mm}Bachelor of Science in Electrical Engineering (2017)}
\newline Concentrations: Power Electronics, Motor Drives, Large-Scale Power Systems (GPA: 3.67)
\vspace{-3mm}
\newline\break\textbf{Olympic College, Bremerton, WA\hspace{3 mm}|\hspace{3 mm}Associate of Science (2015)}

\end{singlespace}
\end{document}
